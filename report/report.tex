% This is based on "sig-alternate.tex" V1.9 April 2009
% This file should be compiled with V2.4 of "sig-alternate.cls" April 2009
%
\documentclass{report}

\usepackage[english]{babel}
\usepackage{graphicx}
\usepackage{tabularx}
\usepackage{subfigure}
\usepackage{enumitem}
\usepackage{url}

\usepackage{color}
\definecolor{orange}{rgb}{1,0.5,0}
\definecolor{lightgray}{rgb}{.9,.9,.9}
\definecolor{java_keyword}{rgb}{0.37, 0.08, 0.25}
\definecolor{java_string}{rgb}{0.06, 0.10, 0.98}
\definecolor{java_comment}{rgb}{0.12, 0.38, 0.18}
\definecolor{java_doc}{rgb}{0.25,0.35,0.75}

% code listings

\usepackage{listings}
\lstloadlanguages{Java}
\lstset{
	language=Java,
	basicstyle=\scriptsize\ttfamily,
	backgroundcolor=\color{lightgray},
	keywordstyle=\color{java_keyword}\bfseries,
	stringstyle=\color{java_string},
	commentstyle=\color{java_comment},
	morecomment=[s][\color{java_doc}]{/**}{*/},
	tabsize=2,
	showtabs=false,
	extendedchars=true,
	showstringspaces=false,
	showspaces=false,
	breaklines=true,
	numbers=left,
	numberstyle=\tiny,
	numbersep=6pt,
	xleftmargin=3pt,
	xrightmargin=3pt,
	framexleftmargin=3pt,
	framexrightmargin=3pt,
	captionpos=b
}

% Disable single lines at the start of a paragraph (Schusterjungen)

\clubpenalty = 10000

% Disable single lines at the end of a paragraph (Hurenkinder)

\widowpenalty = 10000
\displaywidowpenalty = 10000
 
% allows for colored, easy-to-find todos

\newcommand{\todo}[1]{\textsf{\textbf{\textcolor{orange}{[[#1]]}}}}

% consistent references: use these instead of \label and \ref

\newcommand{\lsec}[1]{\label{sec:#1}}
\newcommand{\lssec}[1]{\label{ssec:#1}}
\newcommand{\lfig}[1]{\label{fig:#1}}
\newcommand{\ltab}[1]{\label{tab:#1}}
\newcommand{\rsec}[1]{Section~\ref{sec:#1}}
\newcommand{\rssec}[1]{Section~\ref{ssec:#1}}
\newcommand{\rfig}[1]{Figure~\ref{fig:#1}}
\newcommand{\rtab}[1]{Table~\ref{tab:#1}}
\newcommand{\rlst}[1]{Listing~\ref{#1}}

% General information

\title{Distributed Systems -- Assignment 3}

% Use the \alignauthor commands to handle the names
% and affiliations for an 'aesthetic maximum' of six authors.

\numberofauthors{3} %  in this sample file, there are a *total*
% of EIGHT authors. SIX appear on the 'first-page' (for formatting
% reasons) and the remaining two appear in the \additionalauthors section.
%
\author{
% You can go ahead and credit any number of authors here,
% e.g. one 'row of three' or two rows (consisting of one row of three
% and a second row of one, two or three).
%
% The command \alignauthor (no curly braces needed) should
% precede each author name, affiliation/snail-mail address and
% e-mail address. Additionally, tag each line of
% affiliation/address with \affaddr, and tag the
% e-mail address with \email.
%
% 1st. author
\alignauthor Student One\\
	\affaddr{ETH ID XX-XXX-XXX}\\
	\email{one@student.ethz.ch}
% 2nd. author
\alignauthor Student Two\\
	\affaddr{ETH ID XX-XXX-XXX}\\
	\email{two@student.ethz.ch}
%% 3rd. author
\alignauthor Student Three\\
	\affaddr{ETH ID XX-XXX-XXX}\\
	\email{three@student.ethz.ch}
%\and  % use '\and' if you need 'another row' of author names
%% 4th. author
%\alignauthor Student Four\\
% 	\affaddr{ETH ID XX-XXX-XXX}\\
% 	\email{four@student.ethz.ch}
%% 5th. author
%\alignauthor Student Five\\
% 	\affaddr{ETH ID XX-XXX-XXX}\\
% 	\email{five@student.ethz.ch}
%% 6th. author
%\alignauthor Student Six\\
% 	\affaddr{ETH ID XX-XXX-XXX}\\
% 	\email{six@student.ethz.ch}
}


\begin{document}

\maketitle

\begin{abstract}
Concisely state (i) which Android device you used, (ii) which tasks you completed and which are working correctly or limited, and (iii) what your specific enhancements are.
\end{abstract}

\section{Introduction}

Use the introduction for background information on the assignment.
\begin{itemize}
  \item Please give an overview of the usage of the Lamport times and of the Vector Clocks.
  \item Also write about your architecture to handle the UDP communication.
\end{itemize}

Use references such as books \cite{hello}, papers and theses \cite{REST}, or specifications \cite{RFC2616} whenever available.
Web sites for documentation \cite{devServices}, tutorials, etc. are a special case.
In a thesis, you would put them as footnotes. At this stage, however, you will only have a few ``real references,'' so we put the Web sites into the bibliography.
Cite every source you used throughout the assignment.

\section{Lamport Timestamps}
\begin{itemize}
  \item Describe shortly, how you designed your application to implement this task. Include 1-2 screenshots of your app.
  \item Highlight the backbone of your implementation (methods) and add a state transition diagram to describe the logic behind the handling of communication with the server. Especially, describe how you designed the \texttt{isDeliverable(...)} method.
  \item Describe the main problem encountered in this task and give an overview of your solution.
\end{itemize}

\section{Vector Clocks}
\begin{itemize}
  \item Did you reuse elements from Task 2? Highlight the main differences to Task 2.
  \item Describe how you designed the \texttt{isDeliverable(...)} method.
  \item Describe the main problem encountered in this task and give an overview of your solution.
\end{itemize}

\section{Discussion}
Please reply to the following questions.
\begin{itemize}
  \item What are the main advantages of using Vector Clocks over Lamport Timestamps?
  \item When exactly are two Vector Clocks causally dependent?
  \item We decided in the exercise that we would not let our applications trigger a tick when receiving a message. What would be the implications of ticking on receive?
  \item Does a clock tick happen before or after the sending of a message. What are the implications of changing this?
  \item Read and assess the paper Tobias Landes - Dynamic Vector Clocks for Consistent Ordering of Events in Dynamic Distributed Applications3 that gives a good overview on the discussed methods. In particular, which problem of vector clocks is solved in the paper?
\end{itemize}


\section{Conclusion}

Give an overall conclusion that summarizes the main challenges you encountered, your lessons learned and how you divided the work load in your team.

% The following two commands are all you need in the
% initial runs of your .tex file to
% produce the bibliography for the citations in your paper.
\bibliographystyle{abbrv}
\bibliography{report}  % sigproc.bib is the name of the Bibliography in this case
% You must have a proper ".bib" file

%\balancecolumns % GM June 2007

\end{document}
